\chapter{Kurzfassung}
Das Verstecken von ausführbaren Dateien in Datendateien ist ein häufig verwendeter Verschweigungsmechanismus der
Malwareauthoren verwenden, um ihren Code in gutartig aussehende Dateien zu verstecken, zum Beispiel unter Verwendung von
\acrshort{VBA}-Makros in Microsoft Word Dokumente.
In diese Bachelorarbeit versuchen wir einen Malwareangriff zu reproduzieren, wo die Malwareauthoren Steganografie
verwendet haben, um ein komprimierter \acrshort{HTA} Payload Dropper in einer \acrshort{PNG} Bilddatei zu verstecken. Diese
\acrshort{PNG} Datei wurde dann mittels einer Windows nativer \acrshort{API} (\acrshort{WIA}) in einer \acrshort{BMP}
Datei umgewandelt, was ungewollt auch den komprimierter \acrshort{HTA} Payload extrahiert hat, was Sicherheitsmaßnahmen
ausgewichen hat und die Exekution von dem Payload Dropper ermöglicht hat.

Unsere Reproduktion von der Malware ist nicht gelungen, wobei nur der fehlerhafte Umwandlungsmechanismus fehlgeschlagen hat. 
Deswegen glauben wir, dass dieser auf dem Windows 10 Betriebssystem gepatcht wurde, auch wenn wir keine Patchnotes die
darüber informieren würden gefunden haben. Alle andere Exekutionsschritte wurden erfolgreich ausgeführt, wobei der
Payload Dropper und eine vorextrahierte Payload (wann der Umwandlungsmechanismus nicht verwendet ist)
ausgeführt wurde. Daher ist die Schlussfolgerung unserer Arbeit, dass aktuelle Systeme nicht mehr anfällig auf die 
spezifische Schwachstelle sind, die zum Entwickeln dieses Angriffs verwendet wird.
\clearpage
