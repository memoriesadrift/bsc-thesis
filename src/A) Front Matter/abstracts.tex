\chapter{Abstract}
Concealing executable files in data files is a common concealment mechanism used by malware authors to hide malicious
code within benign looking files, such as the use of \acrshort{VBA} macros within Microsoft Word document files.
In this thesis we attempt to recreate a malware attack where a malicious actor used steganography to hide a compressed 
malicious \acrshort{HTA} payload dropper within a \acrshort{PNG} image. This \acrshort{PNG} was extracted using a native 
Windows image conversion \acrshort{API} (\acrshort{WIA}) to convert the \acrshort{PNG} image to \acrshort{BMP}, which
unintentionally also extracted the compressed payload dropper, bypassing security mechanisms and allowing the payload
dropper to be executed. 

Our recreation of the malware was unsuccessful, with the conversion mechanism being the only part that faltered, leading
us to the belief that the faulty conversion mechanism has been patched on the Windows 10 operating system, though we 
were unable to find any patch notes confirming this. 
All the other steps of the malware were completed successfully, with the payload dropping mechanism as 
well as the macro executing a pre-extracted payload (bypassing the conversion mechanism) worked as in the original
attack. Thus, the conclusion of our work is that current systems are no longer vulnerable to the specific vulnerability
used to engineer this attack.

