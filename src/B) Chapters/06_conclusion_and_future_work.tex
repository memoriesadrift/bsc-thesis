\chapter{Conclusion and Future Work}
The main goal of our work was to attempt to recreate a unique malware that used an innovative payload extraction method
to avoid detection. This malware appeared in April 2021 and was reported on later that year by Malwarebytes
\cite{jazi-article}. We used this report by Malwarebytes, authored by Senior Threat Analyst Hossein Jazi, as the basis
of our recreation, using the source code they provided and amending it where necessary.

The main finding of our work is that the malware in question \emph{no longer works as described} on current systems. The
main system we tested for was Windows 10, reasoning it to be the most commonly used operating system at the moment, as
well as the operating system targeted in the original attack. 
The specific version of the operating system we tested was Windows 10 Pro 64-bit, build 19044, with the version of 
Microsoft Word tested on being a Microsoft 365 copy, version 2205, Build 15225.20204 Click-to-Run.
Our testing could be improved by including multiple versions of Microsoft Office as well as Windows, but we decided to
limit ourselves to what we viewed as a likely target system. 

One of the key limitations that faced us in the recreation was that the image conversion function at the heart of the
attack was not made available by Malwarebytes. We obtained the conversion function from a third party malware analysis
company (Docguard) who published macro source code of a malicious document they analysed, which matched the malicious
document covered in the Malwarebytes article in virtually all regards, with only a few strings changed.

Though we believe that the faulty image conversion function has been patched, testing it with the function obtained
directly from the source that first reported on it would clear any doubts about it not being patched. Regardless, we 
conclude that the \acrfull{WIA} image conversion functionality present in \acrshort{VBA} \emph{has been patched}.

For ease of recreation and further testing, we provide the source code used in our recreation. We provide both the
macro code and the steganography tools we created for use in generating an image payload for use with the exploit.
These resources are provided strictly for informational and educational use.

Further research into this specific malware would most likely serve to check what older systems are still vulnerable.
Our research as well as any further research into this malware will prove useful if a similar steganography method were
to be found in \acrshort{WIA} in the future. 
We think that further research in the field of steganography would prove more fruitful than focusing on the specific 
exploit used in the malware we analyse in this work. 
Analysing other file formats that use compressed data streams to store data and the conversion mechanisms that exist 
for them can also help uncover further vulnerable file formats or conversion algorithms. 

