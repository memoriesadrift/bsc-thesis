\chapter{Related Work}
% TODO: Write a short introduction
\section{What Is Malware?}
Though defining malware might seem as a simple task, formally defining it has been a difficult open problem in 
computer virology for a long time; the precise reasoning for this stems from the fact that each algorithm or
piece of software can be expressed logically and has certain \emph{intended behaviour}, which often isn't
properly defined \cite{malware-definition}. Kramer and Bradfield posit a logical definition of malware which
we won't fully dedicate ourselves to, but their introduction to the concept without the use of logical language
is worth mentioning. They posit malware as software that causes the actual behaviour of some other software
to differ from its intended behaviour, where this difference stems from an incorrectness in verification or
validation of program behaviour, leading to a defining characteristic of malware being the \emph{causation of incorrectness}
\cite{malware-definition}. 

Moving from this more formal definition to a more informal one, Skoudis defines malware as "[...]a set of instructions that 
run on your computer and make your system do something that an attacker wants it to do \cite{malware-book}". For our
intents and purposes this definition is sufficient, and we will further broaden it to our working definition:
\begin{quote}
  Malware is software maliciously designed to do whatever its author wants, unconstrained by legality, consent or
  permission.
\end{quote}

\subsection{Motivation}
Malicious actors can act in a multitude of ways, with motivations behind their actions grouped into a few overarching
categories. While different classifications exist, we will be basing ours on a classification by Brewster et al.


We will cover different reasons for why actors might act maliciously, but sometimes the creation of malware doesn't have
to be malicious at all. Malware can also be created to showcase a vulnerability and call attention to it, so that the 
security of the system  under attack can be improved without causing any actual damage. These kinds of actors are 
called \emph{white hat hackers} \cite{white-hat-hacking-definition}.


\subsubsection{Ideological}
Attackers motivated by ideology fall into this category. In the taxonomy by Brewster et al. this encompasses the
\emph{political, ideological and informational / promotional} categories, wherein the actors act based on a political
agenda (such as espionage, sabotage or political protest), a held belief (such as the belief in freedom of information)
that views hacking into systems as a necessary act, or the desire to disseminate information and increase public
awareness \cite{brewster-malware-motivation}.

A famous example of a political attack is the Stuxnet worm that targeted Iranian nuclear facilities in the year
2010 \cite{brewster-malware-motivation}. Stuxnet is reported to have been perpetrated by the US and Israeli governments,
though unconfirmed. %TODO: Cite (Beaumont and Hopkins, 2012).
An interesting facet of the Stuxnet worm was the fact that it spread through systems without causing any damage until it
arrived at its designated targets, where it activated to sabotage the target systems. %TODO: cite

While ideological actors in the taxonomy by Brewster et al. are similar to political attackers, they can be
distinguished because the beliefs they hold are personal, such as a protest against something they oppose or their 
religion \cite{brewster-malware-motivation}. Informational / promotional actors are, in our opinion, very similar to
these kinds of actors, with a famous example being Edward Snowden. Snowden is wanted by US authorities on charges of
espionage after stealing and subsequently leaking thousands of documents pertaining to government espionage against its
own citizens \cite{snowden}.

\subsubsection{Commercial}
Attackers that pursue some sort of financial, commercial or economic gain fall into this category. Brewster et al.
distinguish between \emph{financially motivated} actors and \emph{commercially motivated} actors, with the main
distinction being that the financially motivated actors act to gain more directly, while the commercial actors might
act out of additional reasons such as economic or industrial espionage or theft of company secrets or intellectual
property \cite{brewster-malware-motivation}.
We believe these motivations to be sufficiently similar to allow them to be grouped under one umbrella term.

\subsubsection{Personal}
The final category we observe joins together the remaining categories in the studied taxonomy, encompassing motivations
that are directly related to a person's own life. These are distinct from the ideologically motivated actors, as their
actions aren't necessarily driven by ideology, but more so by emotion or way of life. This category encompasses actors
that act emotionally, such as out of anger, boredom or who seek revenge, actors that hack because they find it fun or
challenging, or want validation from peers, or even actors that resort to hacking due to how they choose to live their
life, such as trolls hacking as a means of causing emotional distress to their targets
\cite{malware-motivation-classification, brewster-malware-motivation}.

\subsection{Types of Malware}
Malware comes in many shapes in sizes that have some characteristics in common, while differing on others. The most
basic part that all malware has in common is the \emph{payload}, or what the malware is supposed to do 
\cite[p.~12]{aycock-book}. This could be anything, but it is often understood to mean the malicious activity that 
the malware performs. Another property we consider all malware to have in common is an \emph{attack vector} or how
the malware gains access to the victim's system. We will cover attack vectors separately, but some examples include
social engineering, phishing, drive-by attacks, droppers or abuse of a vulnerability.

Where malware begins to differ are the other properties -- Aycock posits a taxonomy based on the following three 
characteristics, with each type of malware being classified on a scale roughly akin to "yes, no, maybe"
for each property.
\begin{quote}
  \begin{enumerate}
    \item \emph{Self-replicating} malware actively attempts to propagate by creating new copies, or instances, 
      of itself. Malware may also be propagated passively, by a user copying it accidentally, for example, 
      but this isn't self-replication.
    \item The \emph{population growth} of malware describes the overall change in the number of malware instances due to
      self-replication. Malware that doesn't self-replicate will always have a zero population growth, but malware with a
      zero population growth may self-replicate.
    \item \emph{Parasitic} malware requires some other executable code in order to exist. "Executable" in this context 
      should be taken very broadly to include anything that can be executed, such as boot block code on a disk, binary 
      code in applications, and interpreted code. It also includes source code, like application scripting languages,
      and code that may require compilation before being executed.
  \end{enumerate}
  \cite[p.~11-12]{aycock-book}
\end{quote}
Where these three characteristics are not sufficient to differentiate two types of malware, additional clarification can
be provided to distinguish the two; for example while \emph{spyware} and \emph{adware} both are not self-replicating,
have no population growth and are not parasitic, their payloads differ -- where spyware collects information for
exfiltration, adware often uses collected information for advertising purposes, spamming the user with advertisements
or exfiltrating information to gain a competitive edge \cite[p.~16-17]{aycock-book}. 

\subsubsection{Viruses}
\subsubsection{Worms}
\subsubsection{Trojan Horses}
\subsubsection{Ransomware}
Ransomware will be discussed in depth in a further section discussing the current threat landscape, as it is an integral
part of it.

\subsection{Concealing Malware}

\subsection{Social Engineering and Phishing}
\subsubsection{Insider Threat}
\subsubsection{Compromised or Weak Credentials}

\section{Current Threat Landscape}
\section{File Format Security}
\subsection{What Is a File}
\subsection{Hiding Extra Content in Files}
\subsubsection{File Format Hacking}
\subsection{Example: Bitmap Image Files}

\section{Microsoft Word Documents}
\subsection{History of the File Format}
\subsubsection{.doc}
\subsubsection{.docx}
\subsection{Macros and Scripting}
\subsection{Use in Malware}

% --------------------------------------------------
% Notes
% --------------------------------------------------

\clearpage

Background
\begin{enumerate}
  \item What is Malware?
  \item Attack Vectors
    \begin{itemize}
      \item Social Engineering
        \begin{itemize}
          \item Phishing / Spear Phishing
          \item Insider Threat
          \item Compromised or Weak Credentials
        \end{itemize}
      \item Vulnerabilities (CVE)
    \end{itemize}
  \item Malware
  \begin{itemize}
    \item Payload types
    \item Concealing / Obfuscation + common techniques
  \end{itemize}
  \item Files
    \begin{itemize}
      \item What is a file
      \item Bitmap images
      \item Hiding extra content in files
      \item File format hacking
    \end{itemize}
  \item Word Documents
    \begin{itemize}
      \item Format description
      \item Macros / Scripting
      \item Use in Malware
      \item What about not using MS Office suite to open documents
    \end{itemize}
  \item Lazarus
    \begin{itemize}
      \item Threat Actor
      \item Advanced Persistent Threat
      \item Actor Profile -- Lazarus
      \item WannaCry?
    \end{itemize}
\end{enumerate}
