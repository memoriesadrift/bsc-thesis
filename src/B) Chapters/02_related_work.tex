\chapter{Related Work}
\section{What Is Malware?} \label{sec:malware}
Though defining malware might seem as a simple task, formally defining it has been a difficult open problem in 
computer virology for a long time; the precise reasoning for this stems from the fact that each algorithm or
piece of software can be expressed logically and has certain \emph{intended behaviour}, which often isn't
properly defined \cite{malware-definition}. 

Kramer and Bradfield posit a logical definition of malware which we won't fully dedicate ourselves to, but their 
introduction to the concept without the use of logical language is worth mentioning. 
They posit malware as software that causes the actual behaviour of some other software to differ from its intended behaviour, 
where this difference stems from an incorrectness in verification or validation of program behaviour, leading to a 
defining characteristic of malware being the \emph{causation of incorrectness} \cite{malware-definition}. 

Moving from this more formal definition to a more informal one, Skoudis defines malware as "[...]a set of instructions that 
run on your computer and make your system do something that an attacker wants it to do \cite{skoudis-book}". For our
intents and purposes this definition is sufficient, and we will further broaden it to our working definition:
\begin{quote}
  Malware is software maliciously designed to do whatever its author wants, unconstrained by correctness, legality, 
  consent or permission.
\end{quote}

\subsection{Motivation}
Malicious actors can act in a multitude of ways, with motivations behind their actions grouped into a few overarching
categories. Although different classifications exist, we will be basing ours on a classification by Brewster et al.
While there are many reasons for malware authors to act maliciously, the creation of malware doesn't always have
to be malicious. Malware can also be created to showcase a vulnerability and call attention to it, so that the 
security of the system  under attack can be improved without causing any actual damage. These kinds of actors are 
called \emph{white hat hackers} \cite{white-hat-hacking-definition}.


\subsubsection{Ideological}
Attackers motivated by ideology fall into this category. In the taxonomy by Brewster et al. this encompasses the
\emph{political, ideological and informational / promotional} categories, wherein the actors act based on a political
agenda (such as espionage, sabotage or political protest), a held belief (such as the belief in freedom of information)
that views hacking into systems as a necessary act, or the desire to disseminate information and increase public
awareness of some issue \cite{brewster-malware-motivation}.

A famous example of a political attack is the Stuxnet worm that targeted Iranian nuclear facilities in the year
2010 \cite{brewster-malware-motivation}. Stuxnet is reported to have been perpetrated by the US and Israeli governments,
though unconfirmed \cite{beaumont-stuxnet}. 
An interesting facet of the Stuxnet worm was the fact that it spread through systems without causing any damage until it
arrived at its designated targets, where it activated to sabotage the target systems \cite{markoff-stuxnet}.

While ideological actors in the taxonomy by Brewster et al. are similar to political attackers, they can be
distinguished because the beliefs they hold are personal, such as a protest against something they oppose or their 
religion \cite{brewster-malware-motivation}. Informational / promotional actors are, in our opinion, very similar to
these kinds of actors, with a famous example being Edward Snowden. Snowden is wanted by US authorities on charges of
espionage after leaking thousands of documents pertaining to government espionage against its citizens \cite{snowden}.

\subsubsection{Commercial}
Attackers that pursue some sort of financial, commercial or economic gain fall into this category. Brewster et al.
distinguish between \emph{financially motivated} actors and \emph{commercially motivated} actors, with the main
distinction being that the financially motivated actors act to gain more directly, while the commercial actors might
act out of additional reasons such as economic or industrial espionage or theft of company secrets or intellectual
property \cite{brewster-malware-motivation}.
We believe these motivations to be sufficiently similar to allow them to be grouped under one umbrella term.

\subsubsection{Personal}
The final category we observe joins together the remaining categories of the studied taxonomy, encompassing motivations
that are directly related to a person's own life. These are distinct from the ideologically motivated actors, as their
actions aren't necessarily driven by ideology, but rather by emotion or a way of life. This category encompasses actors
that act emotionally, such as out of anger, boredom or actors who seek revenge, hack because they find it fun or
challenging, or want validation from peers, or even actors that resort to hacking due to how they choose to live their
life, such as trolls hacking as a means of causing emotional distress to their targets
\cite{malware-motivation-classification, brewster-malware-motivation}.

\subsection{Types of Malware}
Malware comes in many shapes and sizes that have some characteristics in common, while differing on others. The most
basic part that all malware has in common is the \emph{payload}, or what the malware is supposed to do 
\cite[p.~12]{aycock-book}. This could be anything, but it is often understood to mean the malicious activity that 
the malware performs. Another property we consider all malware to have in common is an \emph{attack vector} or how
the malware gains access to the victim's system. Some examples of attack vectors include
social engineering, phishing, drive-by attacks, droppers or abuse of a vulnerability.

Where malware begins to differ are the other properties -- Aycock posits a taxonomy based on the following three 
characteristics, with each type of malware being classified on a scale roughly akin to "yes, no, maybe"
for each property.
\begin{quote}
  \begin{enumerate}
    \item \emph{Self-replicating} malware actively attempts to propagate by creating new copies, or instances, 
      of itself. Malware may also be propagated passively, by a user copying it accidentally, for example, 
      but this isn't self-replication.
    \item The \emph{population growth} of malware describes the overall change in the number of malware instances due to
      self-replication. Malware that doesn't self-replicate will always have a zero population growth, but malware with a
      zero population growth may self-replicate.
    \item \emph{Parasitic} malware requires some other executable code in order to exist. "Executable" in this context 
      should be taken very broadly to include anything that can be executed, such as boot block code on a disk, binary 
      code in applications, and interpreted code. It also includes source code, like application scripting languages,
      and code that may require compilation before being executed.
  \end{enumerate}
  \cite[p.~11-12]{aycock-book}
\end{quote}
Where these three characteristics are not sufficient to differentiate two types of malware, additional clarification can
be provided to distinguish the two; for example while \emph{spyware} and \emph{adware} both aren't self-replicating,
have no population growth and are not parasitic, their payloads differ -- where spyware collects information for
exfiltration, adware often uses collected information for advertising purposes, spamming the user with advertisements
or exfiltrating information to gain a competitive edge \cite[p.~16-17]{aycock-book}. 

\subsubsection{Viruses}
Throughout our research, we have repeatedly come across a definition by Cohen, one of the first 
people to conduct research into computer viruses, which goes as follows: 
\begin{quote}
A virus is a program that can ‘infect’ other programs by modifying them to include a possibly 
evolved version of itself \cite{cohen-virus-course}.
\end{quote}
The term \emph{infect} is understood to mean the modification of the target program to include a 
copy of the virus as explained in the remainder of the definition. 

Describing viruses using the three metrics outlined by Aycock, we consider viruses to self-replicate, have a positive 
population growth and be parasitic \cite[p.~14]{aycock-book}. Though viruses spread across the infected system 
(leading to the positive population growth), they notably don't spread through networks, which is the domain of worms, 
covered in the following section \cite[p.~15]{aycock-book}.

Viruses are generally noted to be structured in three distinct parts: 
\begin{itemize}
  \item the \emph{infection vector} -- how the virus spreads across a system. It doesn't necessarily have to be unique,
    leading to \emph{multipartite} viruses, 
  \item the \emph{trigger}, which is how the virus decides when the payload should be delivered,
  \item and finally the \emph{payload} which is what the virus does, other than just replicate throughout the infected
    system. Usually, the payload intends to cause some sort of damage, or even act as the start of another attack, such
    as the virus payload intending to open a back door to the system, allowing the attacker to access the system and use
    it as part of a botnet.
\end{itemize}
\cite[p.~7, p.~27]{viruses-revealed-book,aycock-book}

Viruses, as described, infect files in order to spread and eventually drop their payload. There are numerous ways this
infection can take place: The file itself can be modified by the virus to contain a copy of the virus in a process known
as overwriting or insertion, the file can be replaced by a copy of the virus file that redirects the user to the
correct file after doing whatever it pleases after the user attempts to open the file (think of it as a more malicious 
shortcut) in what is called a companion virus, or, most importantly for the topic of this work, the virus can be
embedded in the macro section of a document format that supports macros, for example Microsoft Word \cite{aycock-book,
skoudis-book}. 

Macros are a simple tool that was originally intended to increase the productivity of users using word
processors such as Microsoft Word. Using macros allows the user to run arbitrary code at will, for example when the
document is opened, which is easy to weaponise for malicious use. Though the user is often given a warning about the
presence of macros in a document and given the option to run them, these warnings can be easily ignored, or worse, the
user can be deceived into allowing macros to be executed by an attacker.

The payload of a virus can be arbitrary, from having no payload at all and simply infecting more and more files, to
payloads that randomly delete files, clog up memory, create logic bombs or even back doors to the infected system 
\cite{viruses-revealed-book, skoudis-book, cohen-virus-course}. 

\subsubsection{Worms} \label{subsec:worms}
Since viruses spread through files on a system, they are naturally limited in their spread by human interaction
\cite{aycock-book}. In the early days of computing and computer viruses, it was common for users to move floppy
disks between systems, which allowed an easier spread for the virus, since there was a high rate of mobility of physical
disks between computers \cite{cohen-virus-course}. 

Spreading through networks instead of just the local file system is a rather important trait, which is why malware capable of
spreading throughout networks was given a new name: \emph{worm}. Worms share many characteristics with viruses -- on
Aycock's taxonomy they share two properties: they both self-replicate and they both have a positive population growth,
but worms forego being parasitic \cite[p.~15]{aycock-book}. Worms are standalone entities and
don't infect other files or rely on other executables; they can be thought of as infecting the machine itself \cite{
aycock-book, viruses-revealed-book}.

The main draw to worms from a malware author's perspective is the fact that worms often spread at a very rapid pace,
since they don't have to rely on humans in order to propagate \cite{aycock-book, skoudis-book, viruses-revealed-book}.
Though their speed is one draw, their reach is a non-negotiable second -- since they aren't constrained to being passed
around through physical media, they can reach virtually any device in the world, especially in today's interconnected
society \cite{skoudis-book, aycock-book}. This allows attacks to scale well and quickly infect a large amount of systems,
which can, depending on the payload, have potentially devastating consequences.

Another upside is that attacks conducted with worms have the potential to be very difficult to attribute to a particular
attacker, since the worm rapidly spreads throughout the internet, one victim's computer may be infected by the worm from
an IP address located in Ghana, while the next victim might be infected from a Swiss IP, making attribution, persecution
as well as countermeasure development more difficult \cite{skoudis-book}.

Worms are structured similarly to viruses, with a few minor differences. Skoudis proposes a model shaped like a missile,
wherein a worm is described as having a warhead which serves to penetrate the attacked system, a propagation engine,
scanning engine and target selection algorithm to facilitate the propagation of the worm to appropriately chosen and
vulnerable victims, and finally a payload to be dropped on infected devices to compromise the system or otherwise cause
damage to the victim \cite{skoudis-book}.

There are various ways in which a worm can propagate across a network, often being given a name based on the spread 
mechanism, e.g. \emph{email worm}. Because worms are standalone programs and don't rely on any particular files they can
spread through networks through ways that viruses don't, for example by using buffer overflows or underruns to leech
onto open network connections or can even send malicious emails or messages using other messaging clients with infected 
attachments to a victim's contacts \cite{skoudis-book, aycock-book}.

A particularly destructive use of worms is the creation of \emph{botnets}, large distributed networks of computers that
an attacker has established a back door in and compromised to do their bidding. These botnets can then be weaponised to
launch \acrfull{ddos} attacks with overwhelming force \cite{skoudis-book}. Worms are the perfect tool to create these
compromised networks as they spread rapidly across the world and can quickly find new victims and vulnerable machines
while making the resulting attack harder to trace and more powerful \cite{skoudis-book}.

\subsubsection{Trojan Horses}
The name of this malware threat comes from the epic poem \emph{Aeneid} by Virgil, which describes the final siege of the 
war between the Greeks and the Trojans. The Greeks built a large wooden horse and hid a small part of their army inside
it, then pretended to leave. The Trojans thought the horse was left as a gift and dragged it into the city. Unbeknownst
to them, the horse was full of enemy soldiers that snuck out of the horse under the cover of the night and opened the
city gates from the inside for the attacking Greek army, ending the war then and there.

Similarly to the horse the Greeks constructed in the Trojan war, \emph{trojan horses} are malicious programs that
claim to be executing some mundane, harmless task (and they may or may not actually be doing that), while secretly 
executing some malicious task in the background, such as keylogging, or establishing a back door to the victim's system 
\cite[p.~12-13]{viruses-revealed-book, aycock-book}. 
What differentiates them from viruses and worms is that they don't self-replicate and thus don't have a population growth, 
but can be thought of as parasitic, as they perform tasks other programs might, while being malicious in the background 
\cite[p.~12]{aycock-book}. 

In the modern world, the lines between the three blur, however. Crucially, \emph{multipartite} malware relies on the
user launching an application they think will perform some mundane action (as a trojan would) in order to install a
self-replicating worm or virus and trigger the mechanism used to pass it on further \cite{viruses-revealed-book}. This
is just one of many complexities in categorising malware, which we will gloss over for the sake of brevity.

Just like for any other types of malware, a payload is important for the trojan to do any real damage. An important type
of payload often carried by trojans is a backdoor, leading to a so-called \acrfull{rat}, also called Remote Access Trojan 
\cite{viruses-revealed-book}. \acrshort{rat}s allow an attacker (or user of the \acrshort{rat}) to submit commands to
the victimised (target) computer and copy files to and from it, making the machine a tool at the attacker's disposal
\cite{viruses-revealed-book}. The language of the previous sentence is purposefully vague on whether an attacker
is acting maliciously, as remote control (or remote access) software plays a legitimate role in computing -- we often
want to access a device remotely to work or copy files, such as by using \verb+ssh+.

\acrshort{rat}s are a particularly important threat when misused, as they grant an attacker full control over the
victim's computer, turning the victim's device into a \emph{zombie}, potentially being able to use it to fuel
\acrshort{ddos} attacks, as described in the \nameref{subsec:worms} subsection.

\section{Social Engineering and Phishing}
One of the most common attack vectors used by malicious actors is social engineering, 
where the victim is led to perform certain actions, divulge information or grant access to a 
system based on psychological manipulation \cite{advanced-social-engineering-attacks, covid19-social-engineering}. 
Social engineering continues to be among the top threats to companies in 2021, preying on the human
factor and momentary lapses in judgement by individual workers to attack even the most secure of systems
\cite{verizon_2021}. 
The efficiency of these attacks is not hampered by the technical security of a system,
since the people using the system are the weakest link in its security \cite{advanced-social-engineering-attacks,
social-engineering-attacks}.

Though social engineering encompasses a multitude of different types of attacks ranging from physical to
computer-assisted, we will be focusing on the computer-assisted side of social engineering, namely phishing.
Phishing is a sociotechnical social engineering attack usually perpetrated through email or instant messaging 
against a large amount of targets in the hopes of the attack being successful against enough targets to be 
profitable \cite{advanced-social-engineering-attacks}. 

Phishing carries an analogy to fishing -- messages sent to potential victims are analogous to the casting of 
the line of a fishing rod, with the message contents serving as the bait. 
The recipient of the phishing message is led to believe they must take an action outlined in the message,
usually clicking a link or downloading an attachment, which will lead to the malicious actor stealing
personal information from the victim or infecting the device with malware \cite{state-of-phishing}.
Forging convincing looking emails, websites or business documents is much easier than perpetrating a similar 
scheme in real life by, say, opening a fake brick-and-mortar business.
\begin{quote}
  We’ve evolved social and psychological tools over millions of years to help
  us deal with deception in face-to-face contexts, but these are little use to us
  when we’re presented with an email that asks us to do something. \cite{anderson-security-engineering}
\end{quote}

Phishing serves as a very important attack vector in the current day, serving as an attack vector for any of the malware
types described in section \ref{sec:malware}. Exploits that don't require the user to do anything in order to infect
their machine are incredibly dangerous and effective, but more often than not it's easier to convince the victim to
execute a file, or grant a program permissions that it should not have.

The document we analysed, much like many Microsoft Word macro viruses begins with a call to action for the user to
enable macros, something Microsoft Word disables by default, presenting the user with a pop-up asking if they want to
enable macro execution and warning them to only run macros from trusted sources. However, this warning is often not
sufficient, with many users ignoring the warning out of a lapse of judgement, lack of technical skills or knowledge of
what macros can do, or due to social engineering, believing the source to be trusted when it is not
\cite{macro-viruses-users}.

\section{Current Threat Landscape}
An overview of the current threat landscape is crucial to help us understand the role our analysed
attack plays in the overall landscape. Most notably, since infected document files continue to play 
a major role in malware attacks, it's important to understand the currently rising risks that such
files might pose for us, the leading among them being ransomware at this time \cite{enisa_threat_landscape}.

Cybersecurity threats have been on a continuing rise in recent years, both in size and scale 
\cite{enisa_threat_landscape}. Remote work during the COVID-19 pandemic greatly increased
the attack surface for malicious actors, as well as the time needed to detect intrusions 
or system compromise, leading to the average cost of data breaches increasing across the
board \cite{ibm_2020_cost_of_a_data_breach}. The role of remote work in security related 
incidents was further highlighted as the pandemic moved into its later stages in 2021 as
more companies returned to in-office work. In its 2021 "Cost of a Data Breach" report,
IBM reported that security incidents were costlier for companies implementing remote work,
with companies where over 50\% of the workforce worked remotely taking, on average, 58 days
more to identify and contain breaches when compared to companies where less than 50\% of workers
worked remotely \cite{ibm_2021_cost_of_a_data_breach}. This is in line with in the eponymous 
report from 2020, where 70\% of questioned organisations that implemented remote work as a result of the 
COVID-19 pandemic expecting the average cost of data breaches to increase \cite{ibm_2020_cost_of_a_data_breach}. 
The rise in the cost of data breaches continued in the year 2021, rising by a further 10\% to an 
average cost of \$4.24 million \cite{ibm_2021_cost_of_a_data_breach}.

\subsection{Supply Chain Attacks} \label{sec:supply-chain-attacks}
The biggest rising trend of 2021 have been \emph{supply chain attacks}, driven by the rising reliance of 
companies on third party solutions for their IT needs \cite{enisa_threat_landscape, morphisec_threat_landscape}.
The significance of these attacks has been so high, that the \acrfull{ENISA} created a separate threat
landscape report specifically for this threat.
Supply chain attacks are fundamentally comprised of two distinct attacks: the first on a supplier which the 
attackers then use to conduct a second attack on a target which uses the services of the compromised supplier,
be it a customer or another supplier \cite{enisa_supply_chain_threat_landscape}.

While the attack vectors used to conduct the initial attack on the supplier are in line with expectations, 
consisting of for example social engineering, brute force attacks or abuse of vulnerabilities in software
or configurations used by the victim, the attack vector used to conduct the secondary attack is far more
dangerous and highlights the true danger and severity of supply chain attacks.
Because of the customer-supplier relationship between the two victims, there exists an inherent trust
between the two parties that these types of attacks abuse. The danger comes through so-called Trusted Relationship
attacks where a relationship of trust between two parties is compromised by an attacker and used to compromise
the victim's security by using the trusted relationship as a means of lending themselves legitimacy.

In the realms of software this can mean an attacker compromising a supplier's code repositories, infecting
it with malware and then distributing it as an update to the supplier's customers. The customer is lulled into
a false sense of security with the update, as it comes from the supplier, and it appears there is no cause for 
concern. Attacking the supplier and gaining access to their systems can lead to various kinds of abuse 
such as phishing, distributing malware, or impersonating the supplier's personnel \cite{enisa_supply_chain_threat_landscape}.

The attack targets vary slightly between the two attacks, with the supplier attack being perpetrated not just 
in order to conduct the second attack, but also for example to steal code, configurations or even hardware schematics.
By far the most common target for supplier assets is code, with two thirds of the attacks studied by \acrshort{ENISA}
between January 2020 and July 2021 aiming to compromise this asset \cite{enisa_supply_chain_threat_landscape}. 

The attack on the customer carries similar traits with other cyberattacks, the main difference is in the increased 
ease of infiltration dependent on the success of the first attack.
Common targets are for example exfiltrating data, establishing botnets to carry out \acrfull{ddos} attacks or
extorting money from the victim via ransomware. Data exfiltration was the most common between all the attacks
covered in the \acrshort{ENISA} supply chain attack threat landscape with 58\% of attacks targeting this asset
\cite{enisa_supply_chain_threat_landscape}. 

\subsection{Ransomware}
The term ransomware describes a category of malware threats used to digitally extort victims 
by denying access to a device or files, unless a ransom sum is paid, usually in Bitcoin or other 
cryptocurrencies \cite[p.~150]{ransomware_book}. Some ransomware attacks, like WannaCry, additionally 
threaten to delete the victim's files unless the ransom is paid within a certain time limit to further
intimidate victims into paying the ransom. The methods used to infect systems with ransomware are in line
with other common cyberattacks, for example making use of social engineering or exploiting vulnerabilities
\cite{nist_ransomware}.

A worrying trend reported by \acrshort{ENISA} was the increase in use of zero-day vulnerabilities in the
perpetration of ransomware schemes \cite{enisa_threat_landscape}. \emph{Zero-day attacks} are a class of
attacks that exploit vulnerabilities that have not yet been publicly disclosed \cite{zero-day}. They give the 
attacker a massive advantage as it's virtually impossible to defend against them because of their nature -- 
antivirus software has no hash signature to recognise and the software's developers have no chance to patch the
exploit since they aren't aware of it. These kinds of attacks were previously used mainly by \acrfull{APT}s 
and nation-state threat actors, mainly due to their immense value as a free pass to
any target they may wish to attack \cite{enisa_threat_landscape}. The fact that zero-day attacks are becoming more
common in the ransomware sphere means that the high cost of using a zero-day vulnerability is worth it for the 
attackers -- one is led to believe the payouts are high enough to justify it. 

The profitability of ransomware is a large reason for its rise. It's a means of maximising the monetisation of 
malware as attackers become increasingly motivated by financial gains \cite{ransomware-comprehensive, enisa_threat_landscape}.
Some say we are observing a "golden age" of ransomware, as ransomware becomes more available to threat actors
as \acrfull{raas} becomes more and more widespread and attackers target larger targets in search
of higher payouts \cite{enisa_threat_landscape}. Additionally, amoral threat actors are increasingly targeting
vital infrastructure and organisations that rely on access to their data to prevent death of patients
or hold very important legal data, asking for exorbitant ransom fees which they hope the victims will pay in order
not to be complicit in ,for example, the death of a patient \cite[p.~17-18]{morphisec_threat_landscape,
ransomware-book}. 
High profile hacks and payouts continue to motivate these threat actors to try and replicate these successes and 
get a large payout. 

Another increasing trend in ransomware is the utilisation of multiple axes of attack. The result of these multiple
threats has become known as \emph{double extortion ransomware} or even \emph{triple extortion ransomware}. A common
double extortion ransomware attack consists of the encryption of the victim's data alongside its exfiltration,
with the attacker demanding ransom be paid or their files would not only remain decrypted, but would also be leaked
\cite{enisa_threat_landscape, unit42-ransomware}.

As mentioned above, ransomware is distributed much the same as any other malware threat, which is why the developments
in this area are relevant to our topic. Hiding a ransomware payload in a data file is virtually the same as hiding
any other malware in the file. Thus, with ransomware on the rise, we can anticipate payloads of infected Microsoft
Word or PDF documents to deliver ransomware instead of other malware types. In fact, this is already the case -- 
malicious macro-enabled Word documents or exploited PDF files are already among the attack vectors used in ransomware
delivery, often as a form of downloading, de-obfuscating or decrypting the code that takes control of the machine
\cite[p.~8-10]{ransomware-book}.

\subsubsection{Ransomware as a Service (RaaS)}
As was mentioned in subsection \ref{sec:supply-chain-attacks}, companies are increasingly turning to external service providers 
for their IT solutions and the same can be said for threat actors. Though distributing malware with the intent of
letting other actors perpetrate attacks is not a new phenomenon, the \acrfull{raas} business model is surging in
popularity in recent years. Conti and REvil, the two threat actors with the largest profits as well as infections in the ransomware 
space, were both \acrshort{raas} providers, allowing their customers to easily orchestrate ransomware attacks
\cite{enisa_threat_landscape, unit42-ransomware}.

\acrshort{raas} also makes it much easier for inexperienced attackers or script kiddies to get access to ransomware
and carry out attacks -- so long as they have the capital to purchase the ransomware from the dark web
\cite{social-engineering-ransomware-vector}. Additionally, this points to malicious actors growing closer together
as a community, allowing each to focus on their own area of expertise (malware creation, social engineering,
vulnerability discovery, etc.) and sell expertly crafted solutions to other malicious actors for money or even 
a share of the profits of an attack carried out using their tool \cite{enisa_threat_landscape}.
Another shortcut attackers take that has been observed in 2022 by Unit 42 is the purchasing of access to a compromised 
network directly from so-called \emph{initial access brokers} who provide the ransomware attackers with the ability to
directly drop their ransomware into an already compromised network, saving time and money \cite{unit42-ransomware}.
It is easy to see why this kind of relationship is particularly profitable for attackers perpetrating ransomware
attacks. 

Furthermore, \acrshort{raas} comes with an online dashboard for the command and control server, bulk mail spamming
services, or even specialised social engineering teams that further amplify the effectiveness of the attack
\cite{social-engineering-ransomware-vector}. These services, naturally, come at a cost. Some compensation methods
for the \acrshort{raas} creators include a subscription based model, direct payment or a cut of profits
\cite[p.~44-45]{enisa_threat_landscape, ransomware-book}.

Attributing ransomware attacks to specific threat actors is also becoming more difficult due to this model,
as when a certain strain of ransomware is known to, as an example, spread via e-mail in North America,
if that same ransomware then starts spreading via malicious banner ads in Africa, linking the two is difficult
for security researchers, as it is unclear if they belong to the same family of ransomware, or if they are 
entirely different \cite[p.~47]{ransomware-book}. This hampers not only research into ransomware and how it
spreads, but in some cases also recovery efforts. Knowing which strain of ransomware infected a network can
in some cases even help avoid paying ransom altogether if the encryption algorithm used is known to be
crackable.

All these factors have contributed to the rise in ransomware cases, ransom amounts as well as groups
perpetrating these attacks due to \acrshort{raas} lowering the barrier to entry \cite{unit42-ransomware}.
Because of this, we can expect the amount of ransomware groups to continue to rise in the future. Since ransomware
is an increasingly popular attack type, the ability to hide it in infected data files is crucial, or in other words:
being able to understand how ransomware can be hidden within data files is becoming increasingly important.

\subsubsection{Multiple Extortion Ransomware}
Another trend mentioned in passing in subsection \ref{sec:supply-chain-attacks} is the increase in cases where
threat actors choose to pressure the victim on multiple fronts, often two, with three becoming an important
number as well. Double extortion was first observed in late 2019, but is noted to have exploded in popularity
by 2022 with the primary secondary extortion tactic being data exfiltration with the motive of threatening victims
into paying quicker, or demanding additional funds \cite{unit42-ransomware, multiple-extortion-ransomware}. 

This data exfiltration trend arose out of a decreased number of organisations willing to pay ransom due to strong
backups -- due to easy backup solutions like cloud backups more and more companies were able to recover from serious
ransomware attacks by simply resetting their machine to a clean state and restoring a backup, hampering the attackers'
ability to secure the ransom payment \cite{multiple-extortion-ransomware}. Attackers often threaten to publish
the data on the dark web to name and shame the victims into paying the ransom, because even if the 
organisation can recover its data using a back up, the threat of a data leak is much more difficult to stop internally.

An example of a threat actor that perpetrates these kinds of attacks is the Conti threat group. 
The group has been at the forefront of ransomware attacks observed by Unit 42 and has leveraged double extortion
attacks to demand high average ransom payments of \$1.78 million in 2021 and leaking data of over 600 organisations,
including vital infrastructure such as hospitals or law enforcement agencies \cite{unit42-ransomware}.
It is clear that Conti are ruthless attackers, stopping at nothing to secure profits.

Even though double extortion ransomware is still in its infancy, attackers are already innovating on the concept
of using multiple angles of extorting their victims. The term \emph{triple extortion ransomware} was coined mainly
in relation to a ransomware attack on Finnish mental health care provider Vastaamo. The attackers targeted the clinic 
and demanded ransom for decryption of patient data, which they also exfiltrated as part of a double extortion scheme
\cite{kshteri-ransomware}.
However, what set apart this case was that the attackers then additionally extorted the patients, the victim's
customers, by demanding smaller sums of money from them directly, threatening to leak their therapists' 
session notes \cite{vastaamo}.
Ultimately, the company buckled under the heavy financial losses and the data breach and declared bankruptcy,
ceasing operations completely \cite{vastaamo, kshteri-ransomware}.

Attacks like the ones described above have been given the name \emph{multiple extortion ransomware} by Payne and
Mienie to ransomware that allows the attacker multiple opportunities for extorting payments from their victims.
\cite{multiple-extortion-ransomware}.
Payne and Mienie further suggest that the evolution of multiple extortion attacks is an important and ongoing
development in the current threat landscape:
\begin{quote}
  There are many ways in which these kinds of multiple-extortion attacks may already be developing organically,
  with or without forethought. First, third parties who purchase or download a victim's sensitive files may pursue
  additional attempts to blackmail or otherwise extort money from the victim directly using the same information.
  Second, criminals may leverage customer, client, patient or employee data from leaked files to harass,
  intimidate, or extort money from those individuals. Third, and possibly most concerning of all, while an individual
  or organization might not be a valuable or high-profile target at the present time, the negligible cost of storing
  stolen data could mean that many years in the future, the sensitive information of a potential world leader or
  prominent businessperson could be used to blackmail or extort years or possibly decades after an initial attack.
  \cite{multiple-extortion-ransomware} 
\end{quote}
We've already been able to see some of their thoughts become reality during the Vastaamo attack, where the second
development cited can be observed with attackers extorting the victim's customers.

\section{File Format Security}
Files are a fundamental building block of computing, primarily used to store data and index it by a file name.
Different types of files serve different purposes, such as storing images, text, code, or even process 
information (stored in virtual files) on UNIX based computer systems.

A more thorough definition of what a file is, as defined by Silberschatz et al. is as follows:
\begin{quote}
  A file is a named collection of related information that is recorded on secondary storage. From a user's perspective, 
  a file is the smallest allotment of logical secondary storage; that is, data cannot be written to secondary storage 
  unless they are within a file \cite[p.~422]{operating-system-concepts}.
\end{quote}

Silberschatz et al. also appropriately note that the concept of a file is purposefully extremely general, as files can
hold arbitrary data in forms such as alphabetic, numeric or binary; the data can be structured freely like in the case
of text files, or rigidly structured, such as with \acrfull{PNG} files \cite[p.~422]{operating-system-concepts}.

Files themselves are outwardly characterised by a \emph{file name} and \emph{file extension} with the name used to 
identify the file to the user, while the extension is normally used to indicate what the contents of the file may be
\cite[p~.427]{operating-system-concepts}. Files can also contain other data about the file itself, such as a unique
identifier for the file system, information about the file's size, access control information or a timestamp of the
last modification of the file \cite[p.~422]{operating-system-concepts}.

The file extension can also help the user decide what program to use to work with the file in question. 
For example, opening a binary file in a text editor such as \verb+vim+ or \verb+gedit+ will display a garbled mess of
characters, whereas using a specialised program for work with binaries, such as \verb+Ghidra+ will allow the user to
properly work with the file.

Most commonly, files are stored on the disk as a simple array of bytes (in UNIX based systems, for example), with the file 
extension providing hints on how these bytes are to be interpreted, while the operating system often makes no
assumptions as to how these bytes are to be interpreted \cite[p.~428]{operating-system-concepts}. Ultimately, this 
extension can be arbitrary and if, say, a plain text file (\verb+.txt+) would have its extension changed to \verb+.hjkl+ 
the contents of the file would remain unchanged. 

If we used a text editor to open this file it would read the contents correctly, regardless
of the nonsensical extension name. Similarly, changing the extension back to the original \verb+.txt+ would restore every 
sense of normalcy about the file we would expect -- file extension changes made arbitrarily by the user without the use of 
some conversion algorithm will have no effect on the contents of the file itself. 

Notably, however, it's common for operating systems to only allow certain types of operations on certain file types 
(often based on the extension of the file) -- as an example MS-DOS based systems only allow executing files with the 
\verb+.com+, \verb+.exe+ and \verb+.bat+ file extensions, with  \verb+.com+ and \verb+.exe+ being executable files, 
while \verb+.bat+ is a batch file containing commands for the operating system to execute, in ASCII 
\cite[p.~427]{operating-system-concepts}.

\subsection{File Formats}

Though file extensions serve as an indicator to the user, the computer has to know how to parse the raw byte contents of
the file, which is what file format specifications are for. File formats are standards used to encode data into binary 
to be stored in files, as well as to decode it for future use. 

There are many ways in which file formats can be misused to hide malicious content (or intent) from the user, however,
there is also one very simple and quite common way in which file \emph{extensions} can be used to lull the victim into a
false sense of security. The ILOVEYOU virus that circulated in the year 2000 consisted of an email pretending to have a
love letter attached, with the file \verb+LOVE-LETTER-FOR-YOU.TXT.vbs+ attached \cite{viruses-revealed-book}. 

It should be immediately obvious to the reader that this file is not a text file, but instead a \acrfull{vbs} file. When
this file is executed it launches the virus, since the file is a script that can be run on Windows machines
and not a text file as it's masquerading to be. The vulnerability at play, so to speak, is the ability to hide file name 
extensions from being displayed in the GUI of the operating system. In this case, the attackers further tried to mask
the real file extension by displaying the fake extension in upper case letters, relegating the real file extension to
lower case in an attempt to make it evade the victim's eye.

While the above case preyed on the victim not paying attention to the file name extension, there are also ways in which
files can be made to contain content the user would not expect them to. The process of hiding a file inside another
file falls within the definition of steganography, which traditionally concerns itself with hiding messages within other
messages \cite{steganography}.

Using steganography, we can hide arbitrary data within image files, or even audio files, to be retrieved at a later
point by some specialised decoder. A very common method of hiding data within images, mainly concerning the \acrfull{BMP},
\acrfull{GIF} and \acrfull{JPEG} image formats, is hiding the message within the least significant bit of each byte of the image file
\cite{steganography}. By definition, the least significant bit contributes very little to the byte in question, leading
to changes to the least significant bit being difficult if not impossible for the human eye to see \cite{steganography}.

Additionally, it stands to reason that since each bit can only have one or two values, the chance that a bit of our file 
and the least significant bit of the byte we are hiding our bit in are identical is 50\%, so we can expect to have to
change only roughly every other bit when hiding our data. This makes the task of recognising a file as a carrier of a
message hidden by steganographic means even more difficult.

It's also possible to use the file format specifications in creative ways to craft valid files that have some admittedly
strange properties. File format specifications let the program that implements them know how to parse 
a file in the given format, often by indicating what the file is somewhere in the beginning (or close to the beginning)
of the file in a region known as the file header. This file header should appear at the start of the file, but
sometimes, this isn't enforced by specific vendors that chose to implement the standard. 

Such is the case of the Adobe Acrobat Reader which relaxes the criteria for the \acrfull{PDF} file header to not be
required at the start of the file, but instead within the first 1024 bytes of the file \cite{adobe-acrobat-pdf}. 
While this is a deviation from the standard, it has become widespread among other programs and implementations too, 
due to the dominance of Adobe within the market and can lead to space in the file being able to be used for other 
intents and purposes.

Generally speaking, file formats that don't require the header of the format to be located at offset 0, right at the
very beginning of the file are troublesome from the point of view of being able to conceal data, but even more 
troublesome are formats that use a \emph{terminator} character or sequence of characters to signal to the program reading 
the file that the contents of the file are over, even though there may theoretically be further bytes of data following 
the terminator.

\subsection{\acrfull{PNG} File Format}

One notable file format specification that uses a terminator to signal the end of usable data is the \acrfull{PNG}.
The \acrshort{PNG} file format is a thoroughly defined format for storing raster images in a well-compressed, portable
and lossless manner, meant as a replacement for the proprietary, patented \acrfull{GIF} file format \cite{png-standard}.
It's widely used alongside \acrshort{JPEG} for storing image files, boasting an overall higher quality to the
\acrshort{JPEG} format, which uses a lossy compression algorithm to store data. 

The \acrshort{PNG} file format specifies that each valid \acrshort{PNG} file must end with the four byte sequence 
\verb+73 69 78 68+, also called \verb+IEND+ in the specification, which signifies the end of the PNG data stream 
\cite{png-standard}. The use of this terminator means that programs responsible for reading these \acrshort{PNG}s 
will \emph{ignore} all data that comes after this \verb+IEND+ terminator sequence, as they rightfully think the
\acrshort{PNG} data stream has ended, allowing us to use the remaining bytes of the file to store further,
arbitrary data. 

Of course, the intention behind this terminator is clear to us, it was meant to be the end of the file and when it comes
to regular \acrshort{PNG} files it is. However, due to the arbitrary nature of files, nothing is stopping us from 
appending more bytes of data to the end of this \acrshort{PNG} file.

This is the exact trick that was used by the Lazarus \acrshort{APT} group in the malicious document that we analyse to
smuggle a malicious payload disguised as a simple \acrshort{PNG} image embedded in a Word document \cite{jazi-article}.
Of course, detecting these hidden payloads is possible, and the attackers can't just attach infected executables to
images anywhere they want, since their traces are often picked up by antivirus software even through this concealment
method.

\acrshort{PNG} stores image data in a compressed manner, using Zlib compression for its data stream \cite{png-standard}. 
Zlib is an abstraction over the Deflate compression algorithm, which uses an LZ77 variant for compression \cite{zlib-tech}. 
Though this is a relatively mundane detail, it's important to mention, as our re-implementation of the malicous document
requires the compression of a malicious payload to be masked in the \acrshort{PNG} data stream.

\subsection{Microsoft Word Documents}
Microsoft Word is a proprietary word processing software which allows the users to write and edit documents in a simple
\acrfull{GUI}. These documents are most commonly stored in files ending in the extensions \verb+.doc+, 
\verb+.docx+, or \verb+.docm+. Though many word processors exist on the market, with a healthy amount of both
proprietary and open-source options available, Microsoft Word is one of the largest players in the market as part of the
Microsoft Office suite. The Microsoft Office suite is one of Microsoft's most successful products, historically being
responsible for the decline of other Word Processor sales for the DOS operating system, due in part to the success of
the Windows operating system \cite{bergin-word-processors}.

The success of Windows led to Microsoft Office having a 90\% market share in 1994, pushing out all other competitors
from the market at that time \cite{bergin-word-processors}. Even though many more alternatives exist now, they all
implement the ability to read and write Microsoft Word documents, since they remain as the largest document formats 
in use today.

Though the intricacies of how Microsoft Word works don't interest us, it suffices to know that Word provides a
\acrshort{GUI} for the user to write documents, format them, as well as attach images, tables and other items to
documents. Additionally, it also supports saving documents to other file formats, for example \acrshort{HTML} files.

\subsubsection{Macros and Scripting}
One of the defining features of the Microsoft Office suite, including Word, is the ability to write macros to automate tasks.
These macros can be used for benign tasks such as automating repetitive tasks, but have the potential to be much more
malicious. Macros are written using the \acrfull{VBA} programming language, letting the user write and execute arbitrary
code. Once the user writes macros, they can be saved in the document and executed at any time. A document containing
macros must be saved using the \verb+.docm+ file extension. 

These macros are a popular vector for malware, as they allow the attacker to store code within a benign looking file,
often called a lure document. Macros can also be set to automatically execute at certain times, for example when the
document is open. They also run in the background, so the victim can be left completely unaware that a malicious program
is executing on their device.

Luckily, macros no longer run automatically when a document is opened. In recent versions of Word, macros do not execute 
unless the user allows the document to execute macros by clicking a button in a pop-up banner atop the document. 
However, this simple warning is quite inefficient with one study finding that 63.9\% of the participants unnecessarily 
enabled macros \cite{macro-viruses-users}.

\subsubsection{Use in Malware}
The first use of Microsoft Word in malware was the WM.Concept virus in 1995, which had no malicious side effects, aside from
copying itself over the master template for documents, making each new document contain a copy of the virus
\cite{malware-evolution}. This virus used an early scripting language designed for Microsoft Word, but even when
Microsoft Word switched to \acrlong{VBA} as its scripting language, the viruses followed.

Word documents are often used as so-called lures, serving to download a malicious payload onto the victim's device,
mostly using social engineering to trick the user into enabling macros and/or editing in order for the macro to be able
to download or otherwise launch the malicious payload \cite{gajek-macros}. These kinds of attacks are commonly used in conjunction with
spear phishing attacks, with targeted e-mails containing malicious documents as attachments sent out to a select few
individuals within the organisation the attacker is targeting \cite{advanced-social-engineering-attacks}.

The ability to attach arbitrary code to something a user may view as benign is of high value to attackers. After all,
the weakest link in the security of any system are the humans operating it, which is why we think that using malicious 
documents impersonating legitimate business-related documents are especially effective in tricking victims.

Malicious Word documents often start by asking the user to enable editing, enable macros, or both \cite{gajek-macros}.
This type of social engineering attack relies on victims trusting the document and/or its author, or even their lack of
technical knowledge. Once the victim allows macro execution, the attacker's script can do whatever it pleases, most
often communicating with a remote server to obtain a further malicious payload, extracting a malicious payload hidden
in the document, or infecting further files in the system with a copy of the malware.

\clearpage

