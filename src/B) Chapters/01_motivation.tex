\chapter{Motivation}
Ever since my first foray into the field of information security I have been fascinated by the ways in which
different threats to an organisation's security arise. From the ways in which data can be obtained by rummaging
through dumpsters where sensitive documents were dumped without being properly destroyed to sophisticated zero-day
exploits used to distribute malware, the topic that particularly caught my interest was the way in which malicious
payloads can be concealed in relatively mundane looking files.

A perfect example of such file was a malicious Microsoft Word document created by the Lazarus Group \acrlong{APT}
distributed to victims in South Korea via spear phishing. The malware itself was hidden within a \acrfull{BMP} file 
that was itself concealed as a \acrfull{PNG} file. This malicious file was extracted to the victim's computer by 
a macro in the macro-enabled Word document, which is also a very interesting part of the infection process. 

This document piqued my interest for a multitude of reasons, chief among which was the interesting mechanism used
to infect the victim's device, which used a quirk of the \acrshort{BMP} file format and a conversion function built into
the \acrfull{VBA} programming language used to write macros that can be embedded in Word Documents. Using this 
functionality allowed the malicious payload to be concealed under many layers of file formats while also keeping
the extraction process quite simple, almost routine. The attack itself was also rather creative, using an interesting
attack vector and custom toolchain typical for this threat actor.

The problem with attacks like this is that since files are all simply a series of bytes in the end, with the
interpretation being governed largely by how the operating system or program interacting with the file interprets 
those underlying bytes. Due to how file formats are defined, some formats are more suitable for attacks than others
and with the multitude of formats supported and used over the decades of computing, it is only inevitable that there
would be a way to misuse some format in some way -- one of which being \acrshort{BMP} in this case. I believe it is 
valuable to inspect this attack to shed light on how potential future attacks could be carried out. % TODO: Reword or sth

Out of interest as well as scientific rigour I will attempt to recreate this malware, foregoing the malicious payload
of course, and analyse its effectiveness. The main questions I will seek to answer are the following:
\begin{itemize}
  \item In what ways can a file hide malicious content?
  \item How can a concealed malicious payload be extracted from a file and executed?
  \item How do the previous questions come together to drop the RAT in the analysed document?
  \item Can the analysed document be recreated? Does the exploit still work?
  \item Does the analysed document avoid detection by antivirus software?
  \item Are common systems still vulnerable?
\end{itemize}

Thus, the primary goal of this work will be to recreate the malicious Microsoft Word document along with the \acrshort{BMP}
payload and secondary mocked \acrfull{EXE} payload without the actual \acrlong{RAT}. Recreating this malware should help
verify the reproducibility of the original postmortem of the attack and its functionality as well as help gain further
insight into how vulnerable current systems are to a similar attack, if at all. Furthermore, this analysis may yield
advice other than the simple adage of not opening macro-enabled documents. Though, of course, this is always the best
protection mechanism against malware using \acrshort{VBA} macros as their attack vector.

To achieve this goal we will create a facsimile of the malicious document as well as the payload it carried.
This recreation will be based on the postmortem report of the attack written by Hossein Jazi. 
The first part of the recreation will be a dummy \acrlong{EXE} containing a simple program that indicates the 
system would have been compromised if the attack was real. This \acrshort{EXE} will then be hidden in the \acrshort{BMP}
the same way as in the original attack and embedded in the macro-enabled Word document, analogous to the attack.
Finally, we will be executing this faux-malicious document inside virtual machines running the Windows operating system
and tracking how it executes in comparison to the original attack. 

The metrics for measuring the success of this experiment will be rather simple -- recreating the attack in its full
scale will be a full success, while failure after at least one part of the attack succeeds will be deemed a partial
success. If the recreated attack is successful, we will further test its functionality on a range of virtual machines
running the currently supported versions of the Windows operating system to see if the attack can be reproduced on all,
or only some versions. We will also keep an eye out on when or whether antivirus software detects the payload.

In summary, recreating this attack can lend insight into how file formats can be misused to carry malicious payloads and
avoid detection while doing so. It also serves as an effort to validate the previous research done of this malware and
make sure the results of that research are reproducible. Furthermore, the alternative setup using a facsimile of the
malicious file may provide additional insight that had previously gone unnoticed.

